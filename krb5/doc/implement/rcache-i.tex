The replay cache functions deal with verifying that AP_REQ's do not
contain duplicate authenticators; the storage must be non-volatile for
the site-determined validity period of authenticators.

Each replay cache has a string {\bf name} associated with it.  The use of
this name is dependent on the underlying caching strategy (for
file-based things, it would be a cache file name).  The
caching strategy should use non-volatile storage so that replay
integrity can be maintained across system failures.

\subsubsection{The krb5_rc_ops structure}
In order to implement a new replay cache type, the programmer should
declare a {\bf krb5_rc_ops} structure, and fill in the elements of the
structure appropriately, by implementing each of the replay cache
functions for the new replay cache type.  

The prefix element specifies the prefix {bf name} of the the new replay
cache type.  For example, if the prefix {\bf name} is ``dfl'', then if the
program calls \funcname{krb5_rc_resolve} with a credential cache name
such as ``dfl:host'', then \funcname{krb5_rc_resolve}
will call the resolve function (as defined by the {\bf krb5_rc_ops}
structure where the prefix element is ``dfl'') and pass it the
argument ``host''.

Before a new replay cache type can be recognized by
\funcname{krb5_rc_resolve}, it must be registered with the Kerberos
library by calling \funcname{krb5_rc_register}.

\begin{verbatim}
typedef struct _krb5_rc_ops {
    char *type;
    krb5_error_code (*init)((krb5_rcache,krb5_deltat));
    krb5_error_code (*recover)((krb5_rcache));
    krb5_error_code (*destroy)((krb5_rcache));
    krb5_error_code (*close)((krb5_rcache));
    krb5_error_code (*store)((krb5_rcache,krb5_donot_replay *));
    krb5_error_code (*expunge)((krb5_rcache));
    krb5_error_code (*get_span)((krb5_rcache,krb5_deltat *));
    char *(*get_name)((krb5_rcache));
    krb5_error_code (*resolve)((krb5_rcache, char *));
} krb5_rc_ops;
\end{verbatim}


\subsubsection{Per-type functions}
The following entry points must be implemented for each type of
replay cache.

\begin{funcdecl}{init}{krb5_error_code}{\funcin}
\funcarg{krb5_rcache}{id}
\funcarg{krb5_deltat}{auth_lifespan}
\end{funcdecl}

Creates/refreshes the replay cache identified by \funcparam{id} and sets its
authenticator lifespan to \funcparam{auth_lifespan}.  If the 
replay cache already exists, its contents are destroyed.

%Errors: permission errors, system errors

\begin{funcdecl}{recover}{krb5_error_code}{\funcin}
\funcarg{krb5_rcache}{id}
\end{funcdecl}
Attempts to recover the replay cache \funcparam{id}, (presumably after a
system crash or server restart).

%Errors: error indicating that no cache was found to recover

\begin{funcdecl}{destroy}{krb5_error_code}{\funcin}
\funcarg{krb5_rcache}{id}
\end{funcdecl}

Destroys the replay cache \funcparam{id}.
Requires that \funcparam{id} identifies a valid replay cache.

%Errors: permission errors.

\begin{funcdecl}{close}{krb5_error_code}{\funcin}
\funcarg{krb5_rcache}{id}
\end{funcdecl}

Closes the replay cache \funcparam{id}, invalidates \funcparam{id},
and releases any other resources acquired during use of the replay cache.
Requires that \funcparam{id} identifies a valid replay cache.

%Errors: permission errors

\begin{funcdecl}{store}{krb5_error_code}{\funcin}
\funcarg{krb5_rcache}{id}
\funcarg{krb5_donot_replay *}{rep}
\end{funcdecl}
Stores \funcparam{rep} in the replay cache \funcparam{id}.
Requires that \funcparam{id} identifies a valid replay cache.

Returns KRB5KRB_AP_ERR_REPEAT if \funcparam{rep} is already in the
cache.  May also return permission errors, storage failure errors.

\begin{funcdecl}{expunge}{krb5_error_code}{\funcin}
\funcarg{krb5_rcache}{id}
\end{funcdecl}
Removes all expired replay information (i.e. those entries which are
older than then authenticator lifespan of the cache) from the cache
\funcparam{id}.  Requires that \funcparam{id} identifies a valid replay
cache.

%Errors: permission errors.

\begin{funcdecl}{get_span}{krb5_error_code}{\funcin}
\funcarg{krb5_rcache}{id}
\funcout
\funcarg{krb5_deltat *}{auth_lifespan}
\end{funcdecl}
Fills in \funcparam{auth_lifespan} with the lifespan of
the cache \funcparam{id}.
Requires that \funcparam{id} identifies a valid replay cache.

\begin{funcdecl}{resolve}{krb5_error_code}{\funcinout}
\funcarg{krb5_rcache}{id}
\funcin
\funcarg{char *}{name}
\end{funcdecl}

Initializes private data attached to \funcparam{id}.  This function MUST
be called before the other per-replay cache functions.

Requires that \funcparam{id} points to allocated space, with an
initialized \funcparam{id{\ptsto}ops} field.

Since \funcname{resolve} allocates memory,
\funcname{close} must be called to free the allocated memory,
even if neither \funcname{init} or
\funcname{recover} were successfully called by the application.

%Returns:  allocation errors.


\begin{funcdecl}{krb5_rc_get_name}{char *}{\funcin}
\funcarg{krb5_rcache}{id}
\end{funcdecl}

Returns the name (excluding the type) of the rcache \funcparam{id}.
Requires that \funcparam{id} identifies a valid replay cache.

